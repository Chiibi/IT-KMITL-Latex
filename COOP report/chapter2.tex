\chapter{รายละเอียดของงานที่ปฏิบัติ}
\label{chapter:related-theory}

\section{ทบทวนวรรณกรรม}

\subsection{กระบวนการรู้จักลูกค้า}
กระบวนการรู้จักลูกค้า (Know Your Customer: KYC) เป็นกระบวนการทางธุรกิจที่ใช้เพื่อยืนยันและระบุตัวตนของลูกค้า รวมทั้งประเมินความเสี่ยงที่อาจทำให้เกิดความเสียหายต่อธรุกิจ แต่เดิมแล้ว กระบวนการนี้ถูกใช้ในธุรกิจประเภทการเงิน เพื่อป้องกันการฟอกเงินและใช้เป็นข้อตกลงของธนาคาร โดยขั้นตอนของกระบวนการอาจเป็นเพียงการยืนยันรหัสผ่าน การยื่นเอกสารประกอบ ไปจนถึงการยืนยันด้วยของมูลชีวมิติ สำหรับในปัจจุบันนี้แล้วกระบวนการนี้ได้ถูกนำไปใช้ในธุรกิจอื่นด้วย เพราะนอกจากจะช่วยควบคุมความเสี่ยงของการเกิดอาชญากรรมแล้ว ด้วยขั้นตอนของกระบวนการทำให้ผู้ประกอบการสามารถเข้าใจลูกค้ามากขึ้น เพื่อประกอบการตัดสินใจและทำให้ดำเนินธุรกิจได้อย่างรอบคอบมากขึ้น

\subsection{การซื้อ และการลงทะเบียนซิมการ์ด}
ในปัจจุบันการหาซื้อซิมการ์ดมีความสะดวกมากขึ้น เนื่องจากผู้ให้บริการจะการกระจายซิมการ์ดออกไป ยังตัวแทนจำหน่ายเพื่อให้เข้าถึงลูกค้าในวงกว้างแทนที่จะจัดจำหน่ายด้วยตนเองเพียงอย่างเดียว ไม่ว่าจะเป็นร้านสะดวกซื้อ หรือตำแทนจำหน่ายรายย่อยที่เห็นได้ตามศูนย์การค้าทั่วไป หลังจากที่ซื้อซิมการ์ดมาแล้วผู้ใช้จะยังไม่สามารถใช้งานได้จนกว่าจะ ลงทะเบียนซิมการ์ดนั้นเพื่อเป็นการแสดงตัวตนตามประกาศของสำนักงาน กสทช. โดยผู้ใช้งานจะสามารถติดต่อทำการลงทะเบียนได้ตามจุดรับบริการของผู้จำหน่ายซิมการ์ดนั้น

\subsection{แอปพลิเคชันลงทะเบียนซิมการ์ด}
แอปพลิเคชันลงทะเบียนซิมการ์ด "2 แชะ อัตลักษณ์ (2 Shots)" เป็นแอปพลิเคชันที่พัฒนาขึ้นสำหรับผู้ให้บริการเครือค่ายโทรศัพท์เพื่อใช้ในการยืนยันและพิสูจน์ตัวตนผู้ใช้งานของทางรัฐบาลเพื่อป้องกันการปลอมแปลงการลงทะเบียนซิมการ์ดโดยจะมีการตรวจสอบข้อมูลอัตลักษณ์ของผู้ใช้งาน ซึ่งจะต้องใช้ควบคู่กับอุปกรณ์อ่านข้อมูลบัตรประชาชน และสแกนบาร์โคดที่มีมาตรฐานเดียวกับทางรัฐบาล สำหรับตัวแทนจำหน่ายรายย่อยซึ่งไม่สามารถจัดหาอุปกรณ์ดังกล่าวซื้อมีราคาสูงได้ ก็สามารถรับลงทะเบียนด้วยการถ่ายรูปบัตรประบัตรชาชนและใบหน้าของผู้ลงทะเบียนและส่งไปตรวจสอบผ่านแอปพลิเคชันได้เช่นกัน อย่างไรก็ตามแอปพลิเคชันดังกล่าวยังจำเป็นที่จะต้องให้ผู้ใช้บริการเดินทางไปลงทะเบียน ณ จุดบริการที่รับลงทะเบียบซึ่งสร้างภาระทั้งในค่าใช้จ่ายและการเดินทาง อีกทั้งยังไม่ปัญหาเรื่องรหัสผ่านที่ใช้เข้าแอปพลิเคชันนี้หลุดลอดออกไปทำให้ผู้ใช้งานบางส่วนสามารถลงทะเบียนซิมการ์ดเองโดยไม่มีพยานบุคคลที่สามเป็นผู้รับรู้

สำหรับ TORPEDO นั้นเป็นการลงทะเบียนในลักษณะเดียวกับของตัวแทนจำหน่ายรายย่อยโดยจะถ่ายรูปบัตรประชาชนและใบหน้าผู้ใช้งาน กระบวนการทั้งหมดจะทำผ่านระบบวิดีโอคอล ดังนั้นลูกค้าสามารถติดต่อลงทะเบียนซิมการ์ดจากที่ใดก็ได้ที่มีสัญญาณอินเทอร์เน็ตโดยไม่จำเป็นที่จะต้องเดินทางมาที่จุดบริการ

\section{แนวคิด และทฤษฏีที่เกี่ยวข้อง}

\subsection{WebSocket}
WebSocket เป็นโพรโทคอลรูปแบบหนึ่งของการติดต่อสื่อสาร ในลักษณะการสื่อสารแบบสองทิศทางในเวลาเดียวกัน (Full Duplex) บนการเชื่อมต่อด้วย Transmission Control Protocol: TCP บน Port 80 จึงทำให้เข้ากันได้กับ HTTP ดังนั้นเพื่อที่จะใช้งาน WebSocket การขอทำการเชื่อมต่อ (handshake) จะใช้ HTTP Upgrade header เพื่อเปลี่ยจาก HTTP ไปเป็น WebSocket ซึ่งจะมี URLs เป็น ws: และ wss: ที่จะสามารถส่งข้อมูลให้ Client ได้โดยไม่จำเป็นต้องมีการร้องขอตราบใดที่ยังคงมีการเชื่อมต่ออยู่ โดยทั่วไปแล้ว WebSocket เป็นโพรโทคอลที่นิยมใช้ในระบบที่มีความต้องการการสื่อสารแบบ Realtime เช่น Chat และ Videocall

\begin{figure}[h]
	\centering
	\includegraphics[width=0.5\textwidth]{WebSocketConnectionDiagram}
	\caption{WebSocket Connection Diagram}
\end{figure}

\subsection{Session Initiation Protocol}
Session Initiation Protocol: SIP เป็นโพรโทคอลสัญญาณที่มีความสามารถในการสร้าง (initiating), ปรับแต่ง (maintaining) และ ยกเลิก (terminating) การสื่อสารแบบ Realtime รวมถึงข้อมูลมัลติมีเดีย เช่น วิดีโอ, เสียงและการส่งข้อความ SIP ยังสามารถปรับเปลี่ยนทีอยู่ (Address) พอร์ต(Port) เพิ่มลดจำนวนผู้ใช้งานและปริมาณการส่งข้อมูลมัลติมีเดียได้ นิยมใช้ในแอปพลิเคชันหลายประเภท เช่น Videocall, Streaming, Instant messaging เป็นต้น

SIP ถูกออกแบบให้ทำงานในลักษณะ text base เช่นเดียวกับ HTTP อยู่บน Application layer โดยไม่คำนึงถึงชนิดของ Transport layer ทำให้สามารถทำได้ได้ทั้งบน Transmission Control Protocol: TCP และ  User Datagram Protocol: UDP โดยปกติแล้วการส่งข้อมูลมัลติมีเดีย SIP จะต้องทำงานร่วมกับโพรโทคอลอื่นเช่น RTP แต่จะใช้ SIP เท่านั้นเป็นตัวเริ่มต้นในการสื่อสาร ในเครือข่ายของ SIP จะมี User Agent: UA ซึ่งก็คือแอปพลิเคชันที่ทำหน้าที่รับส่ง SIP message สำหรับจัดการ Session ของการเชื่อมต่อ โดย UA แต่ละที่จะถูกกำหนดด้วยที่อยู่ในรูปแบบของ SIP RLs โดยมีรูปแบบเป็น sip:<username>@<host address>  เช่น sip:alice001@example.com
ในการส่งสัญญาณเชื่อมต่อด้วย SIP เริ่มต้นด้วยการที่มี UA คนหนึ่งมีการขอการเชื่อมต่อด้วยการใช้ SIP URLs ของอีกฝ่ายคล้ายกับการใช้โทรศัพท์ที่ต้องใส่เบอร์ของอีกฝ่าย จากนั้น เมื่ออีกฝ่ายตอบรับการเชื่อมต่อ  SIP จะเรียกใช้อื่นโพรโทคอลที่ถูกกำหนดไว้ในการสร้าง Session เพื่อใช้ในการส่งข้อมูล จนกว่าฝ่ายใดฝ่ายหนึ่งจะยกเลิกการเชื่อมต่อ


\begin{figure}[h]
	\centering
	\includegraphics[width=0.5\textwidth]{SIPConnectionDiagram}
	\caption{Session Initiation Protocol Connection Diagram}
\end{figure}

\subsection{Angular architecture}
Angular architecture เป็นสถาปัตยกรรมระบบของ Angular ที่เป็น frontend application framework ซึ่งเกิดจากการรวมกันของส่วนประกอบหลายส่วน ดังรูปที่ .. เพื่อช่วยให้นักพัฒนาสามารถจัดการและดูแลโครงสร้างของระบบได้ง่าย และเป็นรูปแบบเดียวกัน

\setcounter{secnumdepth}{4} 
\subsubsection{Module} เป็นส่วนที่จะรวมการทำงานทั้งในส่วนของ Component, Template รวมไปถึงพฤติกรรมและสิ่งที่จำเป็นในการทำงานของ Component ไว้คล้ายกับกล่องใบหนึ่งที่เราจะนำของข้างในไปใช้ (export) หรือจะเพิ่มของลงในกล่อง (import)
\subsubsection{Component} 
เป็นส่วนที่ควบคุมการแสดงผลหรือ View ของ Component นั้น ๆ เป็นส่วนที่จะกำหนดและควมคุมข้อมูล ตัวแปร และ การทำงาน ซึ่งจะมีความสัมพันธ์กับ Template ผ่าน Metadata ที่จะกำหนดสิ่งที่ Component จะจัดการอีกที
\subsubsection{Template, Directives \& Data binding} 
เป็นส่วนที่ทำการแสดงผลโดยใช้ Angular markup ซึ่งเป็น Syntax เฉพาะที่จะปรับแต่งหน้า HTML ก่อนจะถูกแสดงผล โดยมี Directives ซึ่งเป็นตัวกำหนดการทำงาน และการผูกข้อมูล (binding) เข้ากับ Component ซึ่งมี 2 ลักษณะ คือ
\begin{enumerate}
	\item Event binding เป็นการผูกข้อมูลที่ผู้ใช้งานส่งเข้ามาทาง Template เช่น การกดปุ่ม, การพิมพ์ข้อความ, การเลือกตัวเลือก เป็นต้น เพื่อนำไปอัปเดตข้อมูลใน Component
	\item Property binding เป็นการผูกเพื่อที่จะแทรกข้อมูลที่เกิดจากการเปลี่ยนแปลงใน Component เช่น การเพิ่มค่าให้กับตัวแปร, การเปลี่ยนเงื่อนไข เป็นต้น เพื่อนำไปอัปเดต Template

\end{enumerate}
\subsubsection{Service \& Dependency Injection} 
Service เป็นส่วนที่จัดการการทำงาน และข้อมูลที่ไม่ได้เกี่ยวข้องโดยตรงกับ Component และ Template เช่น การดึงข้อมูลจาก API, Routing เป็นต้น และเพื่อที่จะใช้งานข้อมูลจาก Service เหล่านี้จึงต้องส่งมันเข้าไปใน Component โดยผ่านสิ่งที่เรียกว่า Dependency Injection: DI ซึ่งจะจัดการ และตรวจสอบการส่งขอมูลของ Service
\begin{figure}[h]
	\centering
	\includegraphics[width=0.9\textwidth]{NgArchitech}
	\caption{Angular architecture overview}
\end{figure}

\section{เครื่องมือที่ใช้ในการพัฒนา}

\subsection{Visual Studio Code}
Visual Studio Code: VS Code เป็นโปรแกรม source code editor ในหลายภาษา ที่มีความสามารถและเครื่องมือที่ช่วยเหลือในการพัฒนา เช่น การตรวจสอบ syntax, code refactoring, debuging รวมทั้งมี Extension ที่สามารถติดตั้งเพิ่มเพื่อช่วยให้การพัฒนามีความสะดวก รวดเร็ว และลดความผิดพลาด
\subsection{TypeScript}
TypeScript เป็นภาษาสคริปที่ยังคงความสามารถของ ECMA Script 2015 และเพิ่มความสามารถของ Type System ที่จะกำหนดชนิดของข้อมูลตัวแปรได้ ทั้งยังเพิ่มความสามารถในการเขียนโปรแกรมเชิงวัตถุ (Object Oriented Programming: OOP) TypeScript เป็น transpiler ที่จะแปลงโค้ดในภาษาตัวเองไปเป็น Javascript ทำให้สามารถทำงานรวมกันได้
\subsection{Postman}
Postman เป็นเครื่องมือที่ช่วยในการทดสอบการทำงานของ API ได้ในหลายรูปแบบเพื่อตรวจสอบ Request ที่ส่งไปและ Response รับกลับมา อีกทั้งยังสามารถใช้ในการ จำลองข้อมูล (Mock Data)  ที่จะใช้ทดสอบ API ได้เช่นกัน
\subsection{Git}
Git เป็น version control ที่เป็นระบบที่ใช้จัดเก็บ ติดตามและควบคุมการเปลี่ยนแปลงที่เกิดขึ้นกับไฟล์ชนิดใดก็ตาม ช่วยให้การพัฒนางานในทีมเป็นไปอย่างมีระบบ คนในทีมสามารถใช้โค้ดที่เป็นเวอร์ชันล่าสุดตลอดเวลา หรือสามารถแก้ไข และแยกสายการพัฒนาออกได้
\subsection{Jenkins}
Jenkins เป็น Continuous integration tools ซึ่งเป็นกระบวนการที่จะทดสอบ build และจัดการทำอย่างอื่น กับ Source Code ใน Git repository ทำให้ประหยัดเวลาในการที่จะต้องมาทำการทดสอบ หรือ build ด้วยตัวเอง

\section{ลักษณะขั้นตอนการทำงาน}
การปฏิบัติงานในส่วนของทีมพัฒนาโปรเจค TORPEDO นั้นมีการพัฒนางานเป็นวงรอบ ซึ่งมีการแบ่งเป็นวงรอบเรียกว่า Sprint โดย 1 Sprint จะมีระยะเวลา 14 วัน และเพื่อให้ทีมได้รับรู้สถานะของงาน และติดตามความคืบหน้าได้จึงได้มีการใช้รูปแบบบางส่วนของ SCRUM  เข้ามาช่วยโดยมีกิจกรรมหลักๆที่เกิดขึ้นในแต่ละวงรอบดังนี้
\subsection{Sprint planning}
ทีมจะมาคุยกันเพื่อที่จะวางแผนในการจัดสรรงาน (Task) ตาม Sprint backlog ที่ทาง Software Analysis ได้กำหนดขึ้นมาไว้ก่อนแล้ว เมื่อแบ่งงานกันเสร็จก็จะมีการประเมินเวลาในการทำงานให้เหมาะสม โดยงานและการประเมินเวลาของแต่ละงานจะมีการบันทึกลงในSprint task board ของ Taiga ซึ่งเป็น Project management tools ที่ทำให้ทุกคนในทีมสามารถติดตามกิจกรรมต่างๆได้ ตามรูปที่
\subsection{Daily Scrum}
เป็นการบอกความเคลื่อนไหวของงานที่ตนเองได้รับ เพื่อแจ้งความคืบหน้า และแจ้งปัญหาที่ตนเองพบ รวมถึงการแจ้งสิ่งที่เราได้ทำหรือทำเสร็จสิ้นไปในของเมื่อวาน ให้คนในทีมรับรู้ช่วยกันแก้ไข แบ่งงานที่ล่าช้า หรือยากเกินความสามารถไปทำ ซึ่งโดยปกติตามรูปแบบของ SCRUM แล้วจะเป็นการทำ Standup meeting ที่คนในทีมจะต้องลุกขึ้นยืนและพูดคุยกัน แต่การทำ Daily scrum ของที่บริษัทนี้จะใช้การส่งข้อความลงในแอปพลิเคชัน Slack ดังรูปที่
\subsection{Code review}
เป็นการประเมินการเขียนโปรแกรม เพื่อวิเคราะห์ข้อดี ข้อเสีย ของโครงสร้างและการทำงานของโปรแกรมที่ได้เขียนไปตลอด Sprint ที่ผ่านมา เพื่อนำไปปรับปรุงและแก้ไขใน Sprint ถัดไป ในการปฏิบัติจริงกิจกรรมนี้จะทำในช่วง 2-3 สัปดาห์แรกของการปฏิบัติงาน เพื่อให้มีลักษณะการเขียนโปรแกรมที่สอดคล้องกับคนในทีม หลังจากนั้นจึงมีการประเมินบ้างเป็นครั้งคราว
\subsection{Sprint review}
ในช่วงวันสุดท้ายของแต่ละ Sprint จะมีการประชุมเพื่อสรุปสิ่งที่แต่ละคนในทีมได้ทำไป ย้ายงานที่ไม่สำเร็จไปไว้ใน Sprint หน้า และพูดคุยถึงปัญหาที่เกิดตลอด Sprint ที่ผ่านมาเพื่อนำไปปรับปรุงในการทำงานใน Sprint  ถัดไป 
\begin{figure}[h]
	\centering
	\includegraphics[width=0.9\textwidth]{TaigaTask}
	\caption{Sprint task board ใน Taiga}
\end{figure}
\begin{figure}[h]
	\centering
	\includegraphics[width=0.7\textwidth]{SlackMeet}
	\caption{การทำ Daily scrum ใน Slack}
\end{figure}