\chapter{บทสรุป}
\label{chapter:conclusion}
ตลอดระยะเวลาการฝึกสหกิจศึกษา ณ บริษัท เน็กซี่เทคโนโลยี จำกัด เป็นระยะเวลาทั้งสิ้น 19 สัปดาห์ 4 วัน ระบบยืนยันและพิสูจน์ตัวตนลูกค้าผ่านวิดีโอคอล หรือในชื่อโปรเจค TORPEDO ได้พัฒนาขึ้นเสร็จสิ้นตามแผน โดยตัวระบบ AIS ถูกนำไปใช้จริงซึ่งในเบื้องต้นจะสามารถรองรับการลงทะเบียนซิมการ์ดทั้งในระบบรายเดือน และระบบเติมเงิน อีกทั้งยังรองรับการลงทะเบียนยืนยันตัวตนลูกค้า MPAY ซึ่งเป็นหนึ่งในบริการของ AIS

จากการปฏิบัติงานสหกิจศึกษาได้สะท้อนให้เห็นถึงปัญหาทั้งจากการปฏิบัติงานทั้งที่เกิดนักศึกษาเองก็ดี จากปัจจัยภายนอกก็ดี ซึ่งทำนักศึกษาได้เรียนรู้ถึงข้อผิดผลาด และการแก้ไขปรับปรุง โดยเฉพาะการพัฒนาซอฟต์แวร์ที่ไม่เพียงแค่จะพัฒนาให้สำเร็จแต่ยังต้องคำนึงถึงโครงสร้าง ความซับซ้อนของระบบที่จะรองรับการเปลี่ยนแปลง และความง่ายการดูแลรักษาระบบ นอกจากนี้เรียนรู้ถึงทักษะที่จำเป็นทั้งด้านการสื่อสาร การวิเคราะห์ ระเบียบวินัย และความรับผิดชอบ ซึ่งทั้งหมดที่กล่าวมานี้เป็นสิ่งที่จะหล่อหลอมให้นักศึกษาได้มีประสบการณ์และความรู้พร้อมที่จะทำงานในชีวิตจริง
\section{บทวิเคราะห์ SWOT ของนักศึกษา}

\subsection{จุดเด่น}
นักศึกษาสามารถปรับตัวเข้ากับรูปแบบการทำงานของบริษัทได้อย่างรวดเร็ว นอกจากนี้นักศึกษามีทักษะพื้นฐานและรักที่จะแสวงหาเทคนิคและความรู้ใหม่มาประยุกต์ใช้ในงานให้เกิดประสิทธิภาพและมีความรวดเร็วอยู่เสมอ อีกทั้งยังมีความรอบคอบและใส่ใจในรายละเอียดของการพัฒนาทำให้มีข้อผิดผลาดที่น้อย
\subsection{จุดด้อย}
นักศึกษายังขาดทักษะด้านการสื่อสารที่ดี ทำให้สื่อสารความต้องการให้ผู้อื่นเข้าใจได้ยาก และมักจะสนใจในรายละเอียดที่ไม่จำเป็นมากเกินไปกรณีที่เกิดปัญหาบางครั้ง จึงทำให้การงานล้าช้าไปบ้าง
\subsection{โอกาส}
นักศึกษาได้มีส่วนร่วมในการพัฒนางานให้กับบริษัทขนาดใหญ่ ซึ่งมีทั้งความกดดันและความท้าทายของตัวงาน  นอกจากนี้ยังได้เรียนรู้เทคโนโลยีที่กำลังเป็นที่นิยมในตลาด และสายงานนี้ ทำให้นักศึกษาได้รับความรู้และประสบการณ์อย่างเต็มที่
\subsection{อุปสรรค}
การเดินทางไปสถานประกอบการ และที่อยู่อาศัยของนักศึกษามีระยะทางค่อนข้างไกล ทำให้นักศึกษาเสียเวลาในเดินทางนาน และด้วยวัฒนธรรม และรูปแบบการทำงานของสถานประกอบการที่ค่อนข้างให้อิสระกับพนักงาน ทำให้ต้องมีระเบียบวินัย และความรับผิดชอบในการจัดการตัวเองที่มากขึ้น

\section{ปัญหาและข้อเสนอแนะ}

\subsection{ด้านตัวนักศึกษา}
มีทักษะด้านการสื่อสารที่ไม่ดี ทำให้สื่อสารความต้องการหรือแสดงถึงปัญหาได้ไม่ชัดเจน

\textbf{ข้อเสนอแนะ} ฝึกการแสดงความคิดเห็น และเข้าสังคมพูดคุยกับคนอื่นให้มากขึ้น
\subsection{ด้านมหาวิทยาลัย}
ระบบจัดการสหกิจศึกษา ไม่สมบูรณ์และขาดความชัดเจนทำให้นักศึกษาและสถานประกอบการมีการกรอกเอกสารซ้ำซ้อนหลายครั้งทั้งในระบบออนไลน์ และระบบแยกของแต่ละคณะ ตัวอย่างเอกสารที่ไม่เป็นมาตรฐาน และการแจ้งข้อมูลกับนักศึกษาและสถานประกอบการขาดดความชัดเจน

\textbf{ข้อเสนอแนะ} ควรพัฒนาระบบจัดการสหกิจศึกษาให้สมบูรณ์ ควรให้ข้อมูลแกนักศึกษาและสถานประกอบการจากแหล่งที่มาเดียวกัน มีตัวอย่างเอกสารและรูปเล่มที่ถูกต้องเป็นตัวอย่าง
\subsection{ด้านสถานประกอบการ}
สถานประกอบการมีโปรเจคที่มีขนาดใหญ่ และจำนวนน้อยทำให้มีตัวเลือกในการมอบหมายงานที่มีขนาดเหมาะสม และสอดคล้องกับเงื่อนไขของสหกิจให้กับนักศึกษาได้น้อย

\textbf{ข้อเสนอแนะ} สถานประกอบการควรเตรียมงานที่เหมาะสมกับนักศึกษา