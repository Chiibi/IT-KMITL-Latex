\documentclass{itkmitlcoop}

\usepackage{afterpage}
\usepackage{graphicx,amsmath,latexsym,amssymb,amsthm}
\usepackage{indentfirst}
\usepackage{cite}

\graphicspath{ {images/} }

%1. Your thesis title (THAI)
\newcommand{\ThesisTiTle}{ชื่อโปรเจ็กต์จบ}
%2. Your thesis title (ENG)
\newcommand{\ThesisTiTleENG}{Project name (English)}
%3. Your name
\newcommand{\AuName}{ชื่อ นามสกุล}
%4. Your name ENG
\newcommand{\AuNameENG}{Name Surname}
%5. Your student ID
\newcommand{\SId}{รหัสประจำตัว}
%6. Your advisor
\newcommand{\Advisor}{ชื่อที่ปรึกษา}
%7. Your advisor employee
\newcommand{\Exami}{ชื่อพนักงานที่ปรึกษา}
%8. สถานประกอบการ
\newcommand{\Company}{ชื่อบริษัท}
%9. ภาคเรียนที่ (in normal letters)
\newcommand{\Sem}{1}
%10. ปีการศึกษา (in normal letters)
\newcommand{\AcaY}{2561}
%11. ปีการศึกษา (in normal letters)
\newcommand{\AcaYAD}{2018}
%12. วันส่งรายงาน
\newcommand{\SubD}{10 พฤศจิกายน พ.ศ. 2561}
%13. สาขา
\newcommand{\Department}{วิศวกรรมซอร์ฟแวร์}
%14. วันเริ่มทำงาน
\newcommand{\StartDWork}{23 กรกฎาคม พ.ศ. 2561}
%15. วันสุดท้ายของการทำงาน
\newcommand{\EndDWork}{30 พฤศจิกายน พ.ศ. 2561}
%16. ที่อยู่สถานประกอบการ
\newcommand{\Address}{ที่อยู่}
%17. เว็บไซต์สถานประกอบการ
\newcommand{\Website}{เว็บไซต์}

\begin{document}    
    \frontmatter
    
    \makecover    
    \makeinnercover
    \makeengcover
    \makecopyrightcover
    \makeletter
    \makeack{
        \begin{enumerate}
            \item คุณ ศุภกร ทองตรา  \quad ตำแหน่ง Web Developer, พนักงานที่ปรึษา
            \item คุณ กฤตเมธ ขำคม  \quad ตำแหน่ง Web Developer
        \end{enumerate}
    }
    \makeapproveletter
    \makeabstract{
      \textbf{การยืนยันและพิสูจน์ตัวตน} เป็นส่วนหนึ่งใน \textbf{กระบวนการรู้จักลูกค้า (Know Your Customer)} ซึ่งเป็นกระบวนการสำคัญที่จำเป็นในการทำสัญญาธุรกรรม เพื่อหาข้อเท็จจริงในข้อมูลเพื่อเป็นการลดความเสี่ยงจากเหตุอาชญากรรม หนึ่งในนั้น คือ การลงทะเบียนซิมการ์ด (Sim Card) เนื่องจากปัญหาทั้งในด้านกฏหมาย ความพร้อม และความไม่สะดวกอย่างเช่น การเดินทางไปยังจุดรับบริการที่ห่างไกล การดำเนินงานที่ล้าช้าอีกทั้งผู้ที่มารอรับบริการจำนวนมาก ดังนั้นระบบที่จะช่วยในการลงทะเบียนโดยที่ไม่จำเป็นต้องเดินทางมายังจุดบริการจะช่วยสร้างโอกาสและความพึงพอใจให้กับทั้งผู้ใช้และผู้ให้บริการ ในรายงานฉบับนี้เราได้นำเสนอถึงการประยุกต์ใช้เทคโนโลยีวิดีโอคอล(Videocall) ในการยืนยันและพิสูจน์ตัวตนของลูกค้า แทนการใช้งานระบบ "2 แชะ อัตลักษณ์" ซึ่งเป็นระบบการยืนยันตัวตนของทางรัฐบาล ที่ยังคงมีความซับซ้อนไม่สะดวกต่อการใช้งานและความไม่พร้อมของทางภาครัฐ โดยระบบใหม่นี้จะมีรูปแบบเป็นเว็บแอปพลิเคชันที่รองรับการใช้งานในทุกอุปกรณ์ทำให้ผู้ใช้งานสามารถลงทะเบียนซิมการ์ดได้โดยไม่จำเป็นต้องเดินทางไปยังจุดบริการ นอกจากนี้สินค้าและบริการอื่นนอกเหนือจากนี้ก็ยังสามารถทำกระบวนการรู้จักลูกค้าผ่านระบบนี้ได้เช่นกัน
    }

    \newpage
    \addcontentsline{toc}{chapter}{สารบัญ}
    \tableofcontents
    
    \newpage
    \addcontentsline{toc}{chapter}{สารบัญตาราง}
    \listoftables    
    
    \newpage
    \addcontentsline{toc}{chapter}{สารบัญภาพ}
    \listoffigures

    \mainmatter
    
    \chapter{บทนำ}
\label{chapter:introduction}

\section{ที่มาและความสำคัญ}

กระบวนการรู้จักลูกค้า (Know Your Customer: KYC) เป็นกระบวนการทางธุรกิจที่ใช้เพื่อยืนยันและระบุตัวตนของลูกค้า รวมทั้งประเมินความเสี่ยงที่อาจทำให้เกิดความเสียหายต่อธรุกิจ แต่เดิมแล้ว กระบวนการนี้ถูกใช้ในธุรกิจประเภทการเงิน เพื่อป้องกันการฟอกเงินและใช้เป็นข้อตกลงของธนาคาร สำหรับในปัจจุบันนี้แล้วกระบวนการนี้ได้ถูกนำไปใช้ในธุรกิจอื่นตามมาตราฐานของกระบวนการ~\cite{Nobody06} เพราะนอกจากจะช่วยควบคุมความเสี่ยงของการเกิดอาชญากรรมแล้ว ด้วยขั้นตอนของกระบวนการทำให้ผู้ประกอบการสามารถเข้าใจลูกค้ามากขึ้น และทำงานได้อย่างรอบคอบมากขึ้น
    \chapter{แนวคิด ทฤษฎีและงานวิจัยที่เกี่ยวข้อง}
\label{chapter:related-theory}

บทที่สอง
    \chapter{วิธีการทดลอง}
\label{chapter:experiment}

บทที่สาม
    \chapter{ผลการปฏิบัติงาน}
\label{chapter:result}

การพัฒนาระบบยืนยันและพิสูจน์ตัวตนลูกค้าผ่านวิดีโอคอลจะช่วยให้การลงทะเบียบซิมการ์ดมีความสะดวกแก้ลูกค้ามากขึ้น ช่วยให้ลูกค้าสามารถลงทะเบียบซิมการ์ดจากที่ใดก็ตามที่มีอินเตอร์เน็ตโดยใช้เพียงสมาร์ทโฟน แท็บเล็ตหรือคอมพิวเตอร์เท่านั้น ซึ่งเป็นการสร้างโอกาสให้กับผู้ให้บริการอีกทั้งการตรวจสอบข้อมูลด้วยบุคคลที่สามซึ่งเป็นพนักงานภายในบริษัททำให้มีความผิดพลาดของข้อมูลที่ส่งไปตรวจสอบน้อยลง ส่วนรหัสการเข้าสู่ระบบของเจ้าหน้าที่ผู้ให้บริการจำเป็นต้องเข้าผ่านระบบภายในของ AIS ทำให้ลดปัญหาการลงทะเบียนซิมการ์ดด้วยตนเองจากรหัสผ่านที่หลุดออกไปได้ อย่างไรก็ระบบนี้จำเป็นต้องมีเจ้าหน้าที่ค่อยให้บริการและดูแลอยู่เสมอซึ่งก็ต้องมีต้นทุนที่เพิ่มเข้ามาในส่วนนี้ด้วย
\section{หน้าจอเว็บแอปพลิเคชันในส่วนของลูกค้า (Client)}

สำหรับเว็บแอปพลิเคชันในส่วนของฝั่งลูกค้าหน้า Mockup ดังรูปที่ \ref{Fig:mockup} เป็นหน้าที่ใช้ในขั้นตอนการพัฒนาและการทดสอบเท่านั้น เมื่อใช้ใน Production หน้านี้จะถูกนำออกไป ส่วนหน้าจอ Intro, Videocall, Thanks ดังรูปที่ \ref{Fig:intro} - \ref{Fig:thanks} จะเป็นหน้าจอที่ลูกค้าจะใช้งานเพื่อทำกระบวนการลงทะเบียนหรือเพื่อดูผลลัพธ์ ซึ่งรองรับการแสดงผลในหลายอุปกรณ์ (Responsive) ดังรูปที่ \ref{Fig:responsive}  และรองรับการแสดงผลในสองภาษา ดังรูปที่  \ref{Fig:lang} 
\begin{figure}[h]
	\centering
	\includegraphics[width=0.8\textwidth]{MockupPage}
	\caption{ตัวอย่างหน้า Mockup}
	\label{Fig:mockup}
\end{figure}
\begin{figure}[h]
	\centering
	\includegraphics[width=0.8\textwidth]{IntroPage}
	\caption{ตัวอย่างหน้า Intro}
	\label{Fig:intro}
\end{figure}
\begin{figure}[h]
	\centering
	\includegraphics[width=0.8\textwidth]{VideocallPage}
	\caption{ตัวอย่างหน้าจอ Videocall}
	 \label{Fig:videocall}
\end{figure}
\begin{figure}[!h]
	\centering
	\subfigure[]{
		\label{Fig:thanks:success}
		\includegraphics[width=0.48\textwidth]{ThanksSuccess}  
	}
	\subfigure[]{
		\label{Fig:thanks:failed}
		\includegraphics[width=0.48\textwidth]{ThanksFailed}  
	}
	\caption{ตัวอย่างหน้า Thanks (ก) ลงทะเบียนสำเร็จ (ข) ลงทะเบียนไม่สำเร็จ}
	\label{Fig:thanks}
\end{figure}
\begin{figure}[!h]
	\centering
	\subfigure[]{
		\label{Fig:responsive:mobile}
		\includegraphics[width=0.26\textwidth]{ResMobile}  
	}
	\quad
	\subfigure[]{
		\label{Fig:responsive:window}
		\includegraphics[width=0.67\textwidth]{ResScreen}  
	}
	\caption{ตัวอย่างการแสดงผลแบบ Responsive (ก) Mobile (ข) Windows}
	\label{Fig:responsive}
\end{figure}
\begin{figure}[!h]
	\centering
	\subfigure[]{
		\label{Fig:lang:thai}
		\includegraphics[width=0.4\textwidth]{LangThai}  
	}
	\qquad
	\subfigure[]{
		\label{Fig:lang:eng}
		\includegraphics[width=0.4\textwidth]{LangEng}  
	}
	\caption{ตัวอย่างการแสดงผลในสองภาษา (ก) ภาษาไทย (ข) ภาษาอังกฤษ}
	\label{Fig:lang}
\end{figure}

\section{เว็บแอปพลิเคชันในส่วนของผู้ให้บริการ (Call center)}
ทางฝั่งเว็บแอปพลิเคชันสำหรับผู้ให้บริการนั้นจะเป็นส่วนสำหรับควบคุม และจัดการเพื่อทำกระบวน การรู้จักลูกค้า เมื่อมีลูกค้าติดต่อเข้ามาระบบจะแสดงรายละเอียดและรายชื่อของลูกค้า ดังรูปที่ \ref{Fig:incoming} \\ถ้าหากเจ้าหน้าที่รับสายจะเข้าสู่หน้าวิดีโอคอล ซึ่งในหน้านี้จะแสดงข้อมูลข้อลูกค้าและมีส่วนจัดการสำหรับเจ้าหน้าที่ เช่น ถ่ายรูป วางสาย อนุมัติ/ปฏิเสธการลงทะเบียน ดังรูปที่ \ref{Fig:adminVideocall}  นอกจากนี้ยังสามารถติดตามสถานะการอัพโหลดข้อมูลของลูกค้าได้ ดังรูปที่ \ref{Fig:upload} ซึ่งข้อมูลเหล่านี้จะถูกนำไปใช้ในการสร้างเอกสารเพื่อเป็นหลักฐานในการลงทะเบียนโดยอัตโนมัติ ดังที่ปรากฏในรูปที่ \ref{Fig:report}
\begin{figure}[h]
	\centering
	\includegraphics[width=0.8\textwidth]{Incoming}
	\caption{ตัวอย่างหน้ารายการลูกค้าที่ติดต่อเข้ามา Incoming list}
	\label{Fig:incoming}
\end{figure}
\begin{figure}[h]
	\centering
	\includegraphics[width=0.8\textwidth]{AdminVideocall}
	\caption{ตัวอย่างหน้าควบคุมของเจ้าหน้าที่ระหว่าง Videocall}
	\label{Fig:adminVideocall}
\end{figure}
\begin{figure}[h]
	\centering
	\includegraphics[width=0.8\textwidth]{UploadProcess}
	\caption{ตัวอย่างรายการสถานะการอัพโหลดข้อมูลของลูกค้า}
	\label{Fig:upload}
\end{figure}
\begin{figure}[h]
	\centering
	\includegraphics[width=0.8\textwidth]{Report}
	\caption{ตัวอย่างเอกสารคำขอใช้บริการโทรศัพท์เคลื่อนที่}
	\label{Fig:report}
\end{figure}
    \chapter{สรุปผล}
\label{chapter:conclusion}

บทที่ห้า
    
    \clearpage
    \addcontentsline{toc}{chapter}{บรรณานุกรม}
    \bibliographystyle{IEEEtran}
    \bibliography{reference}
    
    \startappendix
    \chapter{บันทึกเวลาการปฏิบัติงาน}

\chapter{กิจกรรมระหว่างการปฏิบัติงาน}

\chapter{ประวัติผู้เขียน}
    
\end{document}