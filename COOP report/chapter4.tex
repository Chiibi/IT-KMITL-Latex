\chapter{ผลการปฏิบัติงาน}
\label{chapter:result}

การพัฒนาระบบยืนยันและพิสูจน์ตัวตนลูกค้าผ่านวิดีโอคอลจะช่วยให้การลงทะเบียบซิมการ์ดมีความสะดวกแก้ลูกค้ามากขึ้น ช่วยให้ลูกค้าสามารถลงทะเบียบซิมการ์ดจากที่ใดก็ตามที่มีอินเตอร์เน็ตโดยใช้เพียงสมาร์ทโฟน แท็บเล็ตหรือคอมพิวเตอร์เท่านั้น ซึ่งเป็นการสร้างโอกาสให้กับผู้ให้บริการอีกทั้งการตรวจสอบข้อมูลด้วยบุคคลที่สามซึ่งเป็นพนักงานภายในบริษัททำให้มีความผิดพลาดของข้อมูลที่ส่งไปตรวจสอบน้อยลง ส่วนรหัสการเข้าสู่ระบบของเจ้าหน้าที่ผู้ให้บริการจำเป็นต้องเข้าผ่านระบบภายในของ AIS ทำให้ลดปัญหาการลงทะเบียนซิมการ์ดด้วยตนเองจากรหัสผ่านที่หลุดออกไปได้ อย่างไรก็ระบบนี้จำเป็นต้องมีเจ้าหน้าที่ค่อยให้บริการและดูแลอยู่เสมอซึ่งก็ต้องมีต้นทุนที่เพิ่มเข้ามาในส่วนนี้ด้วย
\section{หน้าจอเว็บแอปพลิเคชันในส่วนของลูกค้า (Client)}

สำหรับเว็บแอปพลิเคชันในส่วนของฝั่งลูกค้าหน้า Mockup ดังรูปที่ \ref{Fig:mockup} เป็นหน้าที่ใช้ในขั้นตอนการพัฒนาและการทดสอบเท่านั้น เมื่อใช้ใน Production หน้านี้จะถูกนำออกไป ส่วนหน้าจอ Intro, Videocall, Thanks ดังรูปที่ \ref{Fig:intro} - \ref{Fig:thanks} จะเป็นหน้าจอที่ลูกค้าจะใช้งานเพื่อทำกระบวนการลงทะเบียนหรือเพื่อดูผลลัพธ์ ซึ่งรองรับการแสดงผลในหลายอุปกรณ์ (Responsive) ดังรูปที่ \ref{Fig:responsive}  และรองรับการแสดงผลในสองภาษา ดังรูปที่  \ref{Fig:lang} 
\begin{figure}[h]
	\centering
	\includegraphics[width=0.8\textwidth]{MockupPage}
	\caption{ตัวอย่างหน้า Mockup}
	\label{Fig:mockup}
\end{figure}
\begin{figure}[h]
	\centering
	\includegraphics[width=0.8\textwidth]{IntroPage}
	\caption{ตัวอย่างหน้า Intro}
	\label{Fig:intro}
\end{figure}
\begin{figure}[h]
	\centering
	\includegraphics[width=0.8\textwidth]{VideocallPage}
	\caption{ตัวอย่างหน้าจอ Videocall}
	 \label{Fig:videocall}
\end{figure}
\begin{figure}[!h]
	\centering
	\subfigure[]{
		\label{Fig:thanks:success}
		\includegraphics[width=0.48\textwidth]{ThanksSuccess}  
	}
	\subfigure[]{
		\label{Fig:thanks:failed}
		\includegraphics[width=0.48\textwidth]{ThanksFailed}  
	}
	\caption{ตัวอย่างหน้า Thanks (ก) ลงทะเบียนสำเร็จ (ข) ลงทะเบียนไม่สำเร็จ}
	\label{Fig:thanks}
\end{figure}
\begin{figure}[!h]
	\centering
	\subfigure[]{
		\label{Fig:responsive:mobile}
		\includegraphics[width=0.26\textwidth]{ResMobile}  
	}
	\quad
	\subfigure[]{
		\label{Fig:responsive:window}
		\includegraphics[width=0.67\textwidth]{ResScreen}  
	}
	\caption{ตัวอย่างการแสดงผลแบบ Responsive (ก) Mobile (ข) Windows}
	\label{Fig:responsive}
\end{figure}
\begin{figure}[!h]
	\centering
	\subfigure[]{
		\label{Fig:lang:thai}
		\includegraphics[width=0.4\textwidth]{LangThai}  
	}
	\qquad
	\subfigure[]{
		\label{Fig:lang:eng}
		\includegraphics[width=0.4\textwidth]{LangEng}  
	}
	\caption{ตัวอย่างการแสดงผลในสองภาษา (ก) ภาษาไทย (ข) ภาษาอังกฤษ}
	\label{Fig:lang}
\end{figure}

\section{เว็บแอปพลิเคชันในส่วนของผู้ให้บริการ (Call center)}
ทางฝั่งเว็บแอปพลิเคชันสำหรับผู้ให้บริการนั้นจะเป็นส่วนสำหรับควบคุม และจัดการเพื่อทำกระบวน การรู้จักลูกค้า เมื่อมีลูกค้าติดต่อเข้ามาระบบจะแสดงรายละเอียดและรายชื่อของลูกค้า ดังรูปที่ \ref{Fig:incoming} \\ถ้าหากเจ้าหน้าที่รับสายจะเข้าสู่หน้าวิดีโอคอล ซึ่งในหน้านี้จะแสดงข้อมูลข้อลูกค้าและมีส่วนจัดการสำหรับเจ้าหน้าที่ เช่น ถ่ายรูป วางสาย อนุมัติ/ปฏิเสธการลงทะเบียน ดังรูปที่ \ref{Fig:adminVideocall}  นอกจากนี้ยังสามารถติดตามสถานะการอัพโหลดข้อมูลของลูกค้าได้ ดังรูปที่ \ref{Fig:upload} ซึ่งข้อมูลเหล่านี้จะถูกนำไปใช้ในการสร้างเอกสารเพื่อเป็นหลักฐานในการลงทะเบียนโดยอัตโนมัติ ดังที่ปรากฏในรูปที่ \ref{Fig:report}
\begin{figure}[h]
	\centering
	\includegraphics[width=0.8\textwidth]{Incoming}
	\caption{ตัวอย่างหน้ารายการลูกค้าที่ติดต่อเข้ามา Incoming list}
	\label{Fig:incoming}
\end{figure}
\begin{figure}[h]
	\centering
	\includegraphics[width=0.8\textwidth]{AdminVideocall}
	\caption{ตัวอย่างหน้าควบคุมของเจ้าหน้าที่ระหว่าง Videocall}
	\label{Fig:adminVideocall}
\end{figure}
\begin{figure}[h]
	\centering
	\includegraphics[width=0.8\textwidth]{UploadProcess}
	\caption{ตัวอย่างรายการสถานะการอัพโหลดข้อมูลของลูกค้า}
	\label{Fig:upload}
\end{figure}
\begin{figure}[h]
	\centering
	\includegraphics[width=0.8\textwidth]{Report}
	\caption{ตัวอย่างเอกสารคำขอใช้บริการโทรศัพท์เคลื่อนที่}
	\label{Fig:report}
\end{figure}