\chapter{บทนำ}
\label{chapter:introduction}

\section{ที่มาและความสำคัญ}

กระบวนการรู้จักลูกค้า (Know Your Customer: KYC) เป็นกระบวนการทางธุรกิจที่ใช้เพื่อยืนยันและระบุตัวตนของลูกค้า รวมทั้งประเมินความเสี่ยงที่อาจทำให้เกิดความเสียหายต่อธรุกิจ แต่เดิมแล้ว กระบวนการนี้ถูกใช้ในธุรกิจประเภทการเงิน เพื่อป้องกันการฟอกเงินและใช้เป็นข้อตกลงของธนาคาร สำหรับในปัจจุบันนี้แล้วกระบวนการนี้ได้ถูกนำไปใช้ในธุรกิจอื่นด้วย เพราะนอกจากจะช่วยควบคุมความเสี่ยงของการเกิดอาชญากรรมแล้ว ด้วยขั้นตอนของกระบวนการทำให้ผู้ประกอบการสามารถเข้าใจลูกค้ามากขึ้น และทำงานได้อย่างรอบคอบมากขึ้น ธุรกิจโทรคมนาคมซึ่งถือเป็นอีกหนึ่งธุรกิจที่กระบวนการรู้จักลูกค้าเข้ามามีบทบาทมากขึ้น เราปฏิเสธไม่ได้เลยว่าทุกวันนี้มีอาชญากรรมที่เกิดขึ้นผ่านทางระบบโทรคมนาคมมากขึ้นไม่ว่าจะเป็น การหลอกหลวงผ่านทางโทรศัพท์เคลื่อนที่ที่หลอกให้เหยื่อโอนเงินหรือเพื่อเป้าหมายอื่นๆ ไปจนถึงการใช้โทรศัพท์เคลื่อนที่เป็นตัวจุดชนวนระเบิดเพื่อก่อความไม่สงบ เนื่องความง่ายในการจัดหาและจัดซื้อของซิมการ์ดที่ทุกวันนี้สามารถหาซื้อได้ในร้านสะดวกซื้อทั่วไป อีกทั้งการซื้อขายยังขาดกระบวนการตรวจสอบถึงแหล่งที่มา

อย่างไรก็ตามเพื่อที่จะป้องกันการก่อการร้าย ทางคณะกรรมการกิจการกระจายเสียง กิจการโทรทัศน์ และกิจการโทรคมนาคมแห่งชาติ (กสทช.) ได้มีการออกประกาศ ~\cite{NBTCAnnouncement} บังคับใช้กับผู้ให้บริการเครือข่ายโทรศัพท์ให้จดทะเบียนซิมการ์ดกับผู้ใช้บริการโดยพัฒนาระบบ ``2 แชะ อัตลักษณ์" ให้ผู้จำหน่ายรวมทั้งร้านค้ารายย่อยสแกนบาร์โค้ดของซิมการ์ดและถ่ายรูปบัตรประจำตัวประชาชน รวมทั้งวันที่และร้านค้าที่จำหน่าย เพื่อให้สามารถติดตามตัวในกรณีเกิดเหตุขึ้นได้ ซึ่งก็พบปัญหาจากการที่ร้านค้ารายย่อยส่งข้อมูลที่ไม่ถูกต้อง เช่น ใช้ภาพอื่นที่ไม่เกี่ยวข้อง ใช้ชื่อปลอม เป็นต้น ต่อมาได้ปรับปรุงวิธีการโดยเพิ่มขั้นตอนของการสแกนลายนิ้วมือ และระบบรู้จำใบหน้าซึ่งบังคับผู้ให้บริการลงทุนซื้ออุปกรณ์สำหรับสแกนลายนิ้วมือและตรวจสอบใบหน้าซึ่งมีราคาสูง จนเกิดก็มีปัญหาที่ไม่สามารถจัดหาอุปกรณ์ได้ ในท้ายที่สุดก็ได้ออกแอปพลิเคชันบนโทรศัพท์ แต่ก็ยังมีปัญหารหัสผ่านสำหรับพนักงานรั่วไหลออกไป ทำให้ผู้ใช้งานลงทะเบียนด้วยตนเองได้ และหากสุดท้ายการลงทะเบียนซิมการ์ดไม่สำเร็จ ลูกค้าก็จะต้องเดินทางไปลงทะเบียนที่จุดบริการอยู่ดี ซึ่งจะเห็นได้ว่าวิธีการดังกล่าวสร้างความลำบากและยุ่งยากเป็นอย่างมาก สร้างภาระทั้งค่าใช้จ่ายและการเดินทาง อีกทั้งยังทำลายโอกาสทางธุรกิจอีกด้วย

ในรายงานฉบับนี้เราต้องการที่จะพัฒนาระบบการยืนยันตัวตนลูกค้าในกระบวนการรู้จักลูกค้าที่ลูกค้าไม่จำเป็นต้องเสียเวลาเดินทางมา ณ จุดบริการ โดยเราได้ประยุกต์เทคโนโลยีวิดีโอคอลกับเว็บแอปพลิเคชันในการช่วยให้ผู้ให้บริการและลูกค้าสื่อสารกัน เพื่อที่จะสามารถลงทะเบียนซิมการ์ดได้ โดยจะส่งข้อมูลพร้อมรูปถ่ายคู่กับบัตรประจำตัวประชาชนตัวไปตรวจสอบกับระบบของทางรัฐบาลต่อไป ลูกค้าจะสามารถรับรู้ผลการลงทะเบียบได้ในทันทีหรือหากลูกค้าลงทะเบียนไม่สำเร็จก็ยังสามารถทำการลงทะเบียนใหม่ผ่านระบบนี้ได้ ระบบนี้เป็นระบบที่พัฒนาขึ้นให้กับบริษัท แอดวานซ์ อินโฟร์ เซอร์วิส จำกัด (AIS) ซึ่งเป็นผู้ให้บริการโทรคมนาคมรายหนึ่ง ดังนั้นผู้ที่รับลงทะเบียนจึงเป็นพนักงานภายในบริษัทเท่านั้น ทำให้ไม่มีปัญหาเรื่องการให้ข้อมูลเท็จจากบุคคลที่สาม นอกจากนี้แล้วเรายังสามารถใช้ระบบเดียวกันนี้ในการทำกระบวนการรู้จักลูกค้ากับสินค้าและบริการอื่นได้อีกเช่นกัน

\section{วัตถุประสงค์การปฏิบัติงาน}

\begin{enumerate}
	\item พัฒนาระบบยืนยันและพิสูจน์ตัวตนลูกค้า โดยประยุกต์ใช้เทคโนโลยีในการติดต่อสื่อสารระหว่างผู้ใช้บริการ และผู้ให้บริการ ผ่านอินเทอร์เน็ต
	\item เพื่อเพิ่มพูนประสบการณ์ ความรู้ เทคนิคและการแก้ปัญหาจากการปฏิบัติงานจริง
\end{enumerate}

\section{ขอบเขตของการปฏิบัติงาน}

\begin{enumerate}
	\item พัฒนาระบบยืนยันและพิสูจน์ตัวตนลูกค้า
	\item พัฒนา และแก้ไขปัญหาระบบอื่นที่บริษัทมอบหมายให้รับผิดชอบ
	\item ศึกษาวิธีการและเทคนิคที่มีประสิทธิภาพเพื่อประยุกต์ใช้กับงาน
\end{enumerate}
